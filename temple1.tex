% This template is prepared by UM-SJTU JI SSTIA
% October, 2023

% Environment Setup (Need to be edited only when adding usepackages)
\documentclass{Setup}


% Head Setting (Change the problem and the team number)
\lhead{Problem A}
\chead{The University Physics Competition, Team 111}


\begin{document}


% Title Setting (Change the title, team number and the problem)
\title{Title} 
\author{Team 111, Problem A}
\date{\today} 
\maketitle 
\thispagestyle{fancy}


% Abstract 
\begin{center}
  \large\textbf{Abstract}
\end{center}

\lipsum[1] % Delete it to write your abstract text

\textbf{Key Words: A, B, C} % Change your key words


% Contents
\newpage
\tableofcontents
\newpage


% Introduction
\section{Introduction}
\subsection{Background}
\lipsum[1] % Delete it to write the background of the problem

\subsection{Problem Restatement}
\lipsum[1] % Delete it to write the restatement of the problem


% Assumptions
\section{Assumptions}
Assumptions are acknowledged when they are made throughout the paper, but a list
of key assumptions is also provided here.

\begin{itemize} % List your assumptions here
    \item Assumption A
    \item Assumption B
    \item Assumption C
    \item Assumption D
\end{itemize}

% Notations
\newpage
\section{Notations}

\begin{table}[h!] % List your notations here
\begin{tabularx}{\textwidth}{cc}
\toprule
\textbf{Symbols} & \textbf{Description}\\
\toprule
  $v_{\perp}$ & Component of asteroid's velocity in direction perpendicular to the Earth \\
  $\rho$ & something \\
  $c$ & something \\
\bottomrule
\label{Notation Table}
\end{tabularx}
\end{table}

Here the main notations are defined while their specific values will be discussed and given later.


% Models: the core part of your project(Don't just use Model1, Model2 as the title of the models)
\section{Model 1}
\subsection{SubModel 1}
    % insert a figure
    \begin{figure}[h!]
        \centering
        \includegraphics[width=1.0\textwidth]{Figures/SSTIA.jpg}
        \caption{SSTIA} \label{SSTIA}
    \end{figure}
    
    \lipsum[1-4] % Delete it to write your content
    
\subsection{SubModel 2}
\lipsum[1-4] % Delete it to write your content

\section{Model 2(If need)}
\lipsum[1-7] % Delete it to write your content

\newpage 
% Sometimes the former page is completed and the figure is important. You can put it on one new page.
\begin{figure}[h!]
    \centering
    \includegraphics[width=1.0\textwidth]{Figures/Nice Image.png}
    \caption{Nice Image} \label{Nice Image}
\end{figure}

\newpage
\lipsum[1-4] % Delete it to write your content

\section{Results Discussion}
\lipsum[1-3] % Delete it to write your content


% Model Evaluation: analyze the strengths and weaknesses of your model 
\section{Model Evaluation}
\subsection{Strengths}
\begin{enumerate}
    \item Strength 1
    \item Strength 2
\end{enumerate}
\subsection{Weaknesses}
\begin{enumerate}
    \item Weakness 1
    \item Weakness 2
\end{enumerate}


% Conclusion of the models you made
\section{Conclusion}
\lipsum[1] % Delete it to write your conclusion


% References
\addcontentsline{toc}{section}{References} 
\begin{thebibliography}{99} % List your references here
  \bibitem{ref1} T.J. Ahrens and A.W. Harris. Deflection and fragmentation of near-earth asteroids. \textit{Nature}, 360(6403):429–433, 1992.
  \bibitem{ref2} Josh Handal, Justyna Surowiec, Michael Buckley (2022) \textit{NASA’s DART Mission Hits Asteroid in First-Ever Planetary Defense Test}, NASA. 
  \bibitem{ref3} \url{https://en.wikipedia.org/wiki/Atmosphere_of_Earth}
\end{thebibliography}


% List the programming source files of your model and remember to name each file properly.
\newpage
\begin{appendices}
  \section{Filename.py}

  Short Description about the Codes.

  \begin{lstlisting}[language=Python]
    # This is a sample for Python
    import numpy as np
    import math
    val=50*np.exp(-5)
  \end{lstlisting}
 
  
  \section{Filename.m}

  Short Description about the Codes.

  \begin{lstlisting}[language=Matlab]
    % This is a sample for Matlab
    [x,y,z] = meshgrid(linspace(-1.3,1.3));
    val=(x.^2+(9/4)*y.^2+z.^2-1).^3-x.^2.*z.^3-(9/80)*y.^2.*z.^3;
    isosurface(x,y,z,val,0)
  \end{lstlisting}
\end{appendices}


\end{document}