%----------------------------------------------------------------------------------
%% MCM/ICM LaTeX Template %%
%% 2024 MCM/ICM           %%
%----------------------------------------------------------------------------------
%不超过25面(包括任何内容)
%字体不小于12号
%摘要十分重要!!!
%如果使用AI工具,就需要在AI使用报告中说明,不计入25面中

\documentclass[12pt]{article}
\usepackage{fontspec} % 字体设置
\usepackage{geometry} % 页面设置
\usepackage{setspace} % 行间距
\usepackage{amsmath,amssymb,amsthm} % 公式
\usepackage{unicode-math} % 公式(一些特殊符号)
\usepackage{tabularx} % 表格
\usepackage{booktabs} % 表格线
\usepackage{graphicx} % 图片
\usepackage{float}  % 图片位置控制(htpd,H)
\usepackage{caption} % 图片标题
\usepackage{fancyhdr} % 页眉页脚
\usepackage{listings} % 代码
\usepackage{xcolor} % 代码颜色高亮
\usepackage{url} % 引用链接
\usepackage{appendix} % 附录
\usepackage{hyperref} % 超链接

\newcommand{\Problem}{ABCDEF} % 哪一题,在这里修改
\newcommand{\Team}{1111111} % 队伍编号,在这里修改


\setmainfont{Arial}  % 将主字体设置为 Arial
\geometry{a4paper,scale=0.8,headheight=15pt} % 页面设置,A4纸,缩放0.8
\setstretch{1.5} % 行间距1.5倍
\setmathfont{Latin Modern Math} % 公式字体
\captionsetup[table]{belowskip=10pt} % 表格标题与表格距离
\hypersetup{
    colorlinks=true,
    linkcolor=black,
    citecolor=black,
    urlcolor=black
} % 超链接颜色

\pagestyle{fancy} % 页眉页脚
\lhead{Team \# \Team}
\rhead{Page \thepage\ of 25} % 注意修改总页数
\cfoot{\empty}
\renewcommand{\headrulewidth}{0.4pt} % 页眉线宽度

% 定义代码高亮部分的颜色(随便调的,可改)
\definecolor{codepink}{RGB}{255, 183, 255}
\definecolor{codePink}{RGB}{255, 60, 255}
\definecolor{codepurple}{RGB}{204, 0, 204}
\definecolor{codeYellow}{RGB}{255, 255, 179}
\definecolor{codeyellow}{RGB}{248, 242, 25}
\definecolor{codeGreen}{RGB}{60, 200, 160}
\definecolor{codegreen}{RGB}{142, 255, 29}
\definecolor{codeBlue}{RGB}{0, 151, 254}
\definecolor{codeblue}{RGB}{133, 235, 255}
\definecolor{codegray}{RGB}{234, 234, 234}
\definecolor{codered}{RGB}{255, 102, 153}
\definecolor{backcolour}{RGB}{242, 242, 235}

% 代码高亮部分的样式设置
\lstdefinestyle{mystyle}{
    backgroundcolor=\color{white},      % 代码块背景色  
    basicstyle=\small,                  % 代码块字体大小(\tiny\scriptsize\footnotesize\small\normalsize\large\Large\LARGE\huge\Huge)
    frame=single,                       % 代码之外显示边框 (none/leftline/topline/bottomline/lines/single/shadowbox)
    framesep=0pt,                       % 设置上下边框与代码块内容的距离
    xleftmargin=0pt,                    % 代码块左侧空白
    xrightmargin=0pt,                   % 代码块右侧空白
    breakindent=75pt,                   % 设置换行后的行的缩进
    rulecolor=\color{black},            % 指定边框的颜色
    commentstyle=\color{codeGreen},     % 注释风格颜色
    keywordstyle=\color{codePink},      % 关键字风格颜色 可以规定关键字为keywords={if, end...}
    stringstyle=\color{codeBlue},       % 字符串风格颜色
    numberstyle=\color{codered},        % 行号风格颜色
    numbers=left,                       % 行号位置为左侧(left\right\none)
    numbersep=5pt,                      % 行号与代码块距离 
    breakatwhitespace=false,            % 是否在空格处换行
    breaklines=true,                    % 是否自动换行
    % prebreak=\raisebox{0ex}[0ex][0ex]{\ensuremath{\hookleftarrow}},  % 换行符号
    keepspaces=true,                    % 是否保留空格
    showspaces=false,                   % 是否强调空格
    showstringspaces=false,             % 是否强调代码中的字符串
    showtabs=false,                     % 是否强调制表符
    tabsize=2,                          % 设置制表符长度为2个字符
    % escapeinside={*},                 % 特定的字符,退出源码模式到LaTeX模式 (例如 escapeinside={\%*}{*)})
    captionpos=b,                       % 设置标题位置为底部(t\b)
}
\lstset{style=mystyle}

\begin{document}

\thispagestyle{empty} % 首页无页眉页脚
\vspace*{-12ex}
\centerline{\begin{tabular}{*3{c}}
	\parbox[t]{0.3\linewidth}{\begin{center}\textbf{Problem Chosen}\\ \Large {\Problem}\end{center}}
	& \parbox[t]{0.3\linewidth}{\begin{center}\textbf{2024\\ MCM/ICM\\ Summary Sheet}\end{center}}
	& \parbox[t]{0.3\linewidth}{\begin{center}\textbf{Team Control Number}\\ \Large {\Team}\end{center}}	\\
	\hline
\end{tabular}}

\vspace*{5ex}
{\centering \Large \textbf{Our Paper's Title} \par}

%Summary Sheet
%----------------------------------------------------------------------------------
{\centering \Large \textbf{Summary} \par}
The abstract is very important, especially the first sentence, and its length should slightly exceed half a page.
\begin{itemize}
    \item Problem repetition and clarification
    \item Assumptions and their rationality
    \item Model design and analysis
    \item Model testing and sensitivity analysis
    \item Strengths and weaknesses of the model
\end{itemize}

\noindent \textbf{Keywords:} A, B, C
%----------------------------------------------------------------------------------


% contents 记得更新页面或者用section指令
%----------------------------------------------------------------------------------
\newpage
\tableofcontents
\newpage
%----------------------------------------------------------------------------------


% Introduction 下面列举了四个小部分,可包含两到三个
%----------------------------------------------------------------------------------
\section{\textbf{Introduction}}
\subsection{\textbf{Problem Background}}

\subsection{\textbf{Problem Restatement}}

\subsection{\textbf{Literature Review}}

\subsection{\textbf{Our Work}}
%----------------------------------------------------------------------------------


% Assumptions and Justifications 假设和对应的证明都要写
%----------------------------------------------------------------------------------
\section{\textbf{Assumptions and Justifications}}

Assumptions are acknowledged when they are made throughout the paper, but a list of key assumptions is also provided here.

\begin{enumerate}
    \item \textbf{Assumption A} 
    \item \textbf{Assumption B}  
    \item \textbf{Assumption C}   
    \item \textbf{Assumption D}
\end{enumerate}
%----------------------------------------------------------------------------------


% Notations 一些文章中用到的符号,可以在这里定义
%----------------------------------------------------------------------------------
\section{Notations}
\begin{table}[h!] % List your notations here
    \caption{Notations used in this paper}
    \begin{tabularx}{\textwidth}{cc}
    \toprule
    \textbf{Symbols} & \textbf{Description}\\
    \midrule
      $v_{\perp}$ & Component of asteroid's velocity in direction perpendicular to the Earth \\
      $\rho$ & something \\
      $c$ & something \\
    \bottomrule
    \label{Notation Table}
    \end{tabularx}
\end{table}
    
Here the main notations are defined while their specific values will be discussed and given later.
%----------------------------------------------------------------------------------


% Model Design and Analysis 模型的建立和分析
%----------------------------------------------------------------------------------
\section{\textbf{The name of model 1}}
\subsection{\textbf{Data Description}}

\subsection{\textbf{The Establishment of Model 1}}

\subsection{\textbf{The Solution of Model 1}}

\section{\textbf{The name of model 2}}

\section{\textbf{The name of model 3}}
%----------------------------------------------------------------------------------


% Sensitivity Analysis 灵敏度分析,改变一个变量,观察结果的变化
%----------------------------------------------------------------------------------
\section{\textbf{Sensitivity Analysis}}
%----------------------------------------------------------------------------------


% Model Evaluation and Further Discussion 模型的评价和进一步讨论,优缺点一定要写
%----------------------------------------------------------------------------------
\section{\textbf{Model Evaluation and Further Discussion}}
\subsection{\textbf{Strengths}}
\begin{enumerate}
    \item Strength 1
    \item Strength 2
\end{enumerate}

\subsection{\textbf{Weaknesses}}
\begin{enumerate}
    \item Weakness 1
    \item Weakness 2
\end{enumerate}

\subsection{\textbf{Further Discussion}}
%----------------------------------------------------------------------------------


% Conclusion 总结
%----------------------------------------------------------------------------------
\section{Conclusion}
%----------------------------------------------------------------------------------


% References 参考文献
%----------------------------------------------------------------------------------
\addcontentsline{toc}{section}{References} 
\begin{thebibliography}{99} % List your references here
    \bibitem{ref1} T.J. Ahrens and A.W. Harris. Deflection and fragmentation of near-earth asteroids. \textit{Nature}, 360(6403):429-433, 1992.
    \bibitem{ref2} Josh Handal, Justyna Surowiec, Michael Buckley (2022) \textit{NASA's DART Mission Hits Asteroid in First-Ever Planetary Defense Test}, NASA. 
    \bibitem{ref3} \url{https://en.wikipedia.org/wiki/Atmosphere_of_Earth}
  \end{thebibliography}
%----------------------------------------------------------------------------------


% Appendices 附录
%----------------------------------------------------------------------------------
% List the programming source files of your model and remember to name each file properly.
\newpage
\begin{appendices}
    \section*{\textbf{Appendices}}

    % 图片
    %\begin{figure}[h!]
    %    \centering
    %    \includegraphics[width=0.95\textwidth]{1.png}
    %    \caption{名称}
    %\end{figure}

    % 一般表格
    \subsection*{Appendix 1}

    Short Description about the Table.

    \begin{table}[h!]
        \caption{example}
        \centering
        \begin{tabular}{|c|c|c|c|c|c|}
        \hline
        A & B & $Y1=A'B'$ & $Y2=AB'$ & $Y3=A'B$ & $Y4=AB$ \\ \hline
        0 & 0 & 1  & 0  & 0  & 0  \\ \hline
        0 & 1 & 0  & 0  & 1  & 0  \\ \hline
        1 & 0 & 0  & 1  & 0  & 0  \\ \hline
        1 & 1 & 0  & 0  & 0  & 1  \\ \hline
        \end{tabular}
    \end{table}

    % 代码
    \subsection*{Appendix 2}

    Short Description about the Codes.

    %\begin{lstlisting}[language=Matlab, firstline=2, lastline=12]{helloworld.m} 直接从已知文件中提取代码
    \begin{lstlisting}[language=Matlab, caption=matlab example]
        % This is a sample for Matlab
        [x,y,z] = meshgrid(linspace(-1.3,1.3));
        val=(x.^2+(9/4)*y.^2+z.^2-1).^3-x.^2.*z.^3-(9/80)*y.^2.*z.^3;
        isosurface(x,y,z,val,0)
    \end{lstlisting}

    \begin{lstlisting}[language=Python, caption=Python example]
        # This is a sample for Python
        import numpy as np
        import math
        val=50*np.exp(-5)
    \end{lstlisting}
     
\end{appendices}
%----------------------------------------------------------------------------------

\end{document}